\documentclass[11pt,a4paper,french]{article}
\usepackage[utf8]{inputenc}
\usepackage{amsmath}
\usepackage{latexsym}
\title{Ondes électromagnétiques dans le vide et ondes planes progressives}
\author{Romain Gille}
\date{\today}
\begin{document}
\maketitle
\newpage
\section{Équations de Maxwell dans le vide de charges et de courants}
\subsection{Équation d'onde}
\begin{align*}
&\vec rot ~\vec rot ~\vec E = \vec grad ~div (\vec E) - \vec \Delta \vec E 
& \vec grad ~div(\vec E) = \vec 0 ~\text{car pas de charges}\\
& \vec rot ~\vec rot ~\vec E = \vec \Delta \vec E  
&\vec rot ~\vec E =  {- \partial \vec B \over \partial t} \\
& - \vec rot {\partial \vec B \over \partial t} \\
& - {\partial \over \partial t}(\vec rot ~\vec B)
& \vec rot ~\vec B = \mu_0 \vec j + \epsilon_0 \mu_0 {\partial \vec E \over \partial t} \\
& - {\partial \over \partial t} (\mu_0 \vec j + \epsilon_0 \mu_0 {\partial \vec E \over \partial t})
& \text{On retire}~\mu_0 \vec j ~\text{car il n'y a pas de courant donc}~\vec j = \vec 0 \\
&  - {\partial \over \partial t} (\epsilon_0 \mu_0 {\partial \vec E \over \partial t}) \\
& - \epsilon_0\mu_0 {\partial^2 \vec E \over {\partial t}^2}
\end{align*}
On a donc 
\fbox{
$\vec \Delta \vec E- \epsilon_0 \mu_0 {\partial^2 \vec E \over {\partial t}^2}$
}
\paragraph{}
On fait la même chose avec $\vec B$ : 
\fbox{
$\vec \Delta \vec B- \epsilon_0 \mu_0 {\partial^2 \vec B \over {\partial t}^2}$
}
$$\vec \Delta \vec S - {1 \over c^2} {\partial^2 \vec S \over {\partial t}^2} = \vec O$$
Où $\vec S$ représente $\vec E$ ou $\vec B$ \\
Où c est homogène à une vitesse \\ \\
C'est une équation d'onde : $c^2 = {1 \over \mu_0 \epsilon_0}$ \\
Elle est valable pour d'autres ondes tels que le son.
\begin{align*}
(\vec \Delta - {1 \over c^2}{\partial^2 \over {\partial t}^2})\vec S &= \vec 0 \\
\Box^2 \vec S^2 &= \vec 0
\end{align*}
Équation de  d'Alembert ($\vec \Box$ : d'Alembertien)
\clearpage
\section{Quelques solutions formelles de l'équation d'onde}
$$\vec \Box^2 \vec S(\vec r, t) = \vec \Delta \vec S(\vec r, t) - {1 \over c^2}{\partial^2 \over {\partial t}^2} \vec S(\vec r, t) = \vec 0$$
\subsection{Onde plane progressive à une dimension}
\paragraph{Définition : }
On dit qu'une onde est plane si, à chaque instant t, la fonction $\vec S(\vec r, t)$ à la même valeur en tout point M appartenant à un plan perpendiculaire à une dimension fixe définie par un vecteur unitaire $\vec n$.
\begin{align*}
\vec \Box^2 \vec S(\vec r, t) &= \vec \Delta \vec S(\vec r, t) - {1 \over c^2}{\partial^2 \over {\partial t}^2} \vec S(\vec r, t) = \vec 0 \\
& = {\partial^2 \over {\partial z}^2} \vec S(\vec r, t) - {1 \over c^2}{\partial^2 \over {\partial t}^2} \vec S(\vec r, t) = \vec 0
\end{align*}
On pose $u_\pm = z _\pm ct$ \\
On veut montrer $S _\pm(u_\pm)$ est solution.
\begin{align*}
{\partial \over \partial z} = {\partial u_\pm \over \partial z}{\partial \over \partial u_\pm} = {\partial \over \partial u_\pm} &\Rightarrow {\partial^2 \over {\partial z}^2} = {\partial ^2 \over {\partial u_\pm}^2}\\
{\partial \over \partial t} = {\partial u_\pm \over \partial t}{\partial \over \partial u_\pm} = _\pm c~{\partial \over \partial u_\pm} &\Rightarrow {\partial^2 \over {\partial t}^2} = c^2 {\partial ^2 \over {\partial u_\pm}^2}
\end{align*}
\begin{align*}
{\partial^2 \vec S _\pm\over {\partial u _\pm}^2} - {1 \over c^2}(c^2 {\partial^2 \vec S _\pm \over {\partial u_\pm}^2}) &= \vec 0 \\
{\partial^2 \vec S_\pm\over {\partial u_\pm}^2} - {\partial ^2 \vec S_\pm\over {\partial u_\pm}^2} &= \vec 0\\
\vec 0 &= \vec 0
\end{align*}
$$S_\pm (u _\pm)~\text{est solution}$$
La solution générale est :
\framebox{$\vec S = \vec S_+ (z +ct) + \vec S_-(z - ct)$}
\subsection{Onde plane de direction quelconque}
$\vec n$ vecteur unitaire du sens de propagation \\
$\vec n(\alpha,\beta,\gamma)$ \\
$\vec n = \alpha \vec u_x + \beta \vec u_y + \gamma \vec u_z$ \\
$\alpha^2 + \beta^2 + \gamma^2 = 1$ \\
Dans le cas particulier précédent où $\vec n$ était le long de $O_z$. \\
$\vec n = \vec u_z ~~~~~ \vec n.\vec r = \vec u_z .z \vec u_z = z$ \\
Le cas général est donné par $\vec S(\vec r, t) = \vec S_+(\vec n.\vec r + ct) + \vec S_-(\vec n.\vec r + ct)$ \\
Les plans ne sont plus perpendiculaire à $\vec u_z$.
\clearpage
\subsection{Structure de l'onde plane progressive}
\paragraph{Définition :}
Donner la structure de l'onde, c'est préciser l'orientation de $\vec E$ et $\vec B$ par rapport à la direction de propagation et donner la relation entre leurs intensités. \\
Prenons une onde qui se propage selon z, pas de dépendance en x et y.
\begin{align*}
& E_x(z, t)\vec u_x \\
\vec E = ~~& E_y(z, t) \vec u_y \\
& E_z(z, t) \vec u_z
\end{align*}
\begin{align*}
& B_x(z, t)\vec u_x \\
\vec B = ~~& B_y(z, t) \vec u_y \\
& B_z(z, t) \vec u_z
\end{align*}
On sait que dans le vide de charges et de courants : \\
$$div \vec E = 0$$
$$div \vec B = 0$$ \\
$${\partial E_x \over \partial x} + {\partial E_y \over \partial y} + {\partial E_z \over \partial z} = 0 \Rightarrow {\partial E_z \over \partial z} = 0$$ \\
$${\partial B_x \over \partial x} + {\partial B_y \over \partial y} + {\partial B_z \over \partial z} = 0 \Rightarrow {\partial B_z \over \partial z} = 0$$
$$E_z = cste~~~~~~~B_z = cste$$
donc
\begin{align*}
& E_x(z, t)\vec u_x \\
\vec E = ~~& E_y(z, t) \vec u_y \\
& \vec 0
\end{align*}
\begin{align*}
& B_x(z, t)\vec u_x \\
\vec B = ~~& B_y(z, t) \vec u_y \\
&\vec 0
\end{align*}
$\vec E$ et $\vec B$ sont dis transverses
\clearpage
\begin{align*}
&{\partial \over \partial x} 
&&E_x 
&&&-{\partial E_y \over \partial z}~~~~~~
&&&& - {\partial B_x \over \partial t}\\
\vec rot~\vec E = -{\partial \vec B \over \partial t}~\Rightarrow~
&{\partial \over \partial y}~~~\wedge 
&&E_y ~~~ = 
&&&{\partial E_x \over \partial z}~~~ =
&&&& -{\partial B_y \over \partial t}\\
&{\partial \over \partial z} 
&& 0
&&& 0~~~~~~~~
&&&& 0\\
\end{align*}
$${\partial E_y \over \partial z} = {\partial Bx \over \partial t}~~\text{et}~~ {\partial E_x \over \partial z} = {\partial B_x \over \partial t}$$
\begin{align*}
u_\pm = z_\pm~ct &\rightarrow u = z-ct \\ \\
{\partial u \over \partial z}{\partial E_y \over \partial u} = -{\partial E_y \over \partial u} &= -{\partial u \over \partial t}{\partial B_x \over \partial u} =-c {\partial B_x \over \partial u}\\ \\
{\partial E_y \over \partial u} = -c {\partial B_x \over \partial u} &\Rightarrow E_y = -cB_x + constante
\end{align*}
On trouve également $E_x = cB_y + constante$
\begin{align*}
& E_x~\vec u_x \\
\vec E = ~~& E_y~\vec u_y \\
& 0
\end{align*}
\begin{align*}
 -&{E_y \over c}~\vec u_x \\
\vec B = ~~ &{E_x \over c}~\vec u_y \\
&0
\end{align*}
$$\vec E . \vec B = -{E_x E_y \over c} + {E_x E_y \over c} = 0$$
Donc $\vec E$ et $\vec B$ Sont perpendiculaires.
\clearpage
\begin{align*}
&||\vec E||^2 = E_x^2 + E_y^2 \\ 
&||\vec B||^2 = {E_y^2 \over c^2} + {E_x^2 \over c^2} = {||\vec E||^2 \over c^2} \\
& \text{donc} \\
&||\vec E|| = c~||\vec B||
\end{align*}
$$\vec E \wedge \vec B = ||\vec E||~||\vec B||~\vec u_z = {||\vec E||^2 \over c}\vec u_z = c~||\vec B||^2 ~\vec u_z \Rightarrow (\vec u_z, \vec E,\vec B)~\text{est un trièdre rectangle direct}$$
\section{Ondes sphériques}
$\vec \Box^2 \vec S(\vec r, t) = \vec 0$ \\
$\vec S(\vec r, t)$ à la même valeur en tout point d'une sphère de centre le point d'émission. \\
Donc $\vec S(\vec r, t)$ ne dépend pas de $\phi~\text{et}~\theta \Rightarrow \vec S(\vec r, t) = \vec S(r, t)$. \\
$\vec \Delta \vec S(\vec r, t) = {1 \over r}{\partial^2 r \vec S(r, t) \over {\partial r}^2}$
\paragraph{}
L'équation devient :
\begin{align*}
{1 \over r}{\partial^2 r \vec S \over {\partial r}^2} - {1 \over c^2}{\partial^2 \vec S \over {\partial t}^2} &= \vec 0 \\
{\partial^2 r \vec S \over {\partial r}^2} - {r \over c^2}{\partial^2 \vec S \over {\partial t}^2} &= \vec 0 \\
{\partial^2 r \vec S \over {\partial r}^2} - {1 \over c^2}{\partial^2 r\vec S \over {\partial t}^2} &= \vec 0 \\
\end{align*}
$\vec \Phi(r, t) = r \vec S(r, t)$ est solution de l'équation. \\
$\vec \Phi(r, t) = \vec F_- (r-ct) + \vec F_+ (r+ct)$ \\
$\vec S(r, t) = {1 \over r}\vec F_- (r-ct) + {1 \over r} \vec F_+ (r+ct)$ \\ \\
${1 \over r}\vec F_- (r-ct)$ est une onde sphérique divergente à partir de l'origine à la vitesse c. \\
${1 \over r} \vec F_+ (r+ct)$ est une onde sphérique convergente vers l'origine à la vitesse c. \\
\end{document}