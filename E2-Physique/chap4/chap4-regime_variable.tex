\documentclass[11pt,a4paper,french]{article}
\usepackage[utf8]{inputenc}
\usepackage{amsmath}
\title{Régime variable}
\author{Romain Gille}
\date{\today}
\begin{document}
\maketitle
\newpage
\section{Equations de Maxwell dans le vide}
\paragraph{}
Vide rempli de charges $\rho(\vec r, t)$ et de courant $\vec j(\vec r, t)$
\subsection{Equation de Maxwell-Faraday}
\paragraph{}
Quelque soit le cas considéré, il se ramène à la variation dans le temps du flux magnétique à travers le circuit contenant le courant induit. \\
De plus, le sens du courant induit est tel que le champ magnétique propre qu'il crée tend à s'opposer à la variation du champ magnétique qui lui a donné naissance : c'est la loi de Lenz. \\
La circulation de la force motrice :
$$e = {1 \over q} \oint_C \vec f_m . \vec dl$$
\subsubsection{Lois de Faraday}
\paragraph{}
Expérimentalement, il a été déterminé que 
$$e = - {d\phi \over dt}~~\text{avec}~~\phi = \iint_S \vec B . \vec dS$$
Force de Lorentz :
$$\vec F_m = q \vec E$$
On utilise la formule de la circulation
$$\vec F_m = q \vec E \Rightarrow e = \oint_C \vec E . \vec dl$$
J'applique le théorème de Stokes-Ampère
$$e = \iint_S \vec rot\vec E . \vec dS$$
$$\iint_S \vec rot \vec E . \vec dS = - {d \over dt} \iint_S \vec B . \vec dS = - \iint_S {\partial \vec B \over \partial}. \vec dS$$
$$\iint_S [\vec rot\vec E . \vec dS + {\partial \vec B \over \partial t} . \vec dS] = 0$$
$$\vec rot \vec E = -{\partial \vec B \over \partial t}~~\text{: Équation de Maxwell-Faraday}$$
\clearpage
\begin{align*}
div(\vec rot \vec E) = 0 &= - div {\partial \vec B \over \partial t} \\
&= - {\partial \over \partial t}(div \vec B) \\
& \Rightarrow div \vec B = f(\vec r)~~~\text{or pas de monopôle magnétique} \\
& \Rightarrow f(\vec r) = 0
\end{align*}
En régime variable aussi : 
$$div\vec B = 0$$
\subsection{Loi de Maxwell-Gauss}
\paragraph{}
Dans le régime variable, la loi de Gauss est toujours valable.
$$div \vec E = {\rho(\vec r, t) \over \epsilon_0}$$
\subsection{Loi de Maxwell-Ampère}
\subsubsection{Loi de conservation de charge}
\begin{align*}
Q(t)&=\iiint_V \rho d \tau \\
{dQ(t) \over dt} &= \iint \vec j . \vec dS \\
{d \over dt} \iiint_V \rho(\vec r, t)d\tau &= -\iiint_V div \vec j .d\tau \\
&\Rightarrow \iiint_V [{\partial \rho (\vec r, t) \over \partial t} + div \vec j]. d\tau = 0 \\
{\partial \rho(\vec r, t) \over \partial t} + div \vec j &= 0 ~~\text{: équation de la continuité }
\end{align*}
\subsubsection{Equation de Maxwell-Ampère}
\paragraph{}
La loi $\vec rot\vec B = \mu_0 \vec j$ n'est plus valable. \\ \\
Par contre $\vec rot\vec B = \mu_0 \vec j + \vec f(\vec r, t)$ est plus général et variable. \\
Déterminons $\vec f(\vec r, t)$ :
\begin{align*}
div(\vec rot \vec B)= 0 &= \mu_0 div \vec j + div \vec f(\vec r, t) \\
&= - \mu_0 {\partial \rho(\vec r, t) \over \partial t} + div \vec f(\vec r, t) \\
&= - \mu_0 {\partial \over \partial t}(\epsilon_0 div \vec E) + div \vec f(\vec r, t) \\
&= - \mu_0 \epsilon_0 div ({\partial \vec E \over \partial t}) + div \vec f(\vec r, t) \\
&= div [-\mu_0 \epsilon_0 { \partial \vec E \over \partial t} + \vec f(\vec r, t)] = 0 \\
&\Rightarrow \vec f(\vec r, t) = \mu_0 \epsilon_0 {\partial \vec E \over \partial t} \\
& \vec rot \vec B = \mu_0 \vec j + \mu_0 \epsilon_0 {\partial \vec E \over \partial t} ~~ \text{ : Maxwell $ \rightarrow $ densité de courant de déplacement}
\end{align*}
$\vec E$ et $\vec B$ forment un ensemble indissociable que l'on nomme le champ électromagnétique.
\paragraph{Propriétés de symétrie du champ électromagnétique}
\subparagraph{Attention : }
L'étude des symétries n'est pas aussi simple que dans les régimes statiques.
Dans le régime variable les termes $\partial \vec B \over \partial t$ et $\partial \vec E \over \partial t$ modifient les symétries.
\end{document}