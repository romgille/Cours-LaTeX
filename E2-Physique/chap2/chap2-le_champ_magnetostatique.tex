\documentclass[11pt,a4paper,french]{article}
\usepackage[utf8]{inputenc}
\usepackage{amsmath}
\title{ESIEE E2 \\ Le champ magnétostatique}
\author{Romain Gille}
\date{\today}
\begin{document}
\maketitle
\newpage
\section{Le courant électrique}
\subsection{Densité volumique de courant}
Dans un référentiel donné, un courant électrique traduit un mouvement d'ensemble de charges électriques. \\
Imaginons un petit volume $d\tau$ centré sur un point P dans lequel des charges en densité $\rho(P)$ se déplacent à la même vitesse $\vec v$. \\
On définit le vecteur densité volumique de courant $\vec j(P)$ par $\vec j(P) = \rho(P)\vec v ~~~en~~~ A.m^{-2}$. \\ \\
Remarque :
\begin{itemize}
\item S'il existe des charges différentes 'i' en densité $\rho_i$ avec une vitesse $\vec v_i$ :
$$\vec j(P) = \sum_i \vec j_i(P) = \sum_i \rho_i(P) \vec v_i$$
\item $\rho(P)$ est la densité de charge qui se déplacent.
\item Si on impose un champ électrique $\vec E_{ext}$ : la force subie par chaque charge est de la forme $\vec F = q \vec E$. \\
Dans un métal, seuls certains électrons sont capables de quitter les atomes d'origine. On les appelle électrons libres. \\
Posons $n_+$ densité de charges supérieure à 0 par unité de volume : 
$\rho_+(P) = n_+e$. \\
Posons $n_-$ densité de charges inférieure à 0 par unité de volume : $\rho_-(P) = n_-e$.
$$n_+ = n_- = n ~~~~~ \rho_+(P)+ \rho_-(P) = ne(1-1) = 0 = \rho(P)$$
Sous l'effet de $\vec E$ supposons $n_i^-$ électrons libres avec une vitesse $$\vec v~~~:~~~\vec j = n_i^-(-e)\vec v$$
L'intensité d'un courant est le nombre de charges qui traversent par unité de temps une surface donnée ($\Sigma_i$)
$$dI = {dq \over dt} = {\rho d\tau \over dt} = {\rho v dt dS \over dt}$$
$\rho$ est la charge contenue dans le tube de courant de longueur vdt et de section dS.
$$I = \iint_\Sigma \vec j.\vec dS ~~ en ~~A.$$
\end{itemize}
\newpage
\subsection{Densité surfacique de courant}
Si le courant volumique se répartit sur une faible épaisseur (cas de conducteur en ruban) on invente une densité surfacique de courant $\vec j_s = \sigma \vec v$ \\($\sigma$ densité surfacique de courant).
$$\vec j_s = \vec j_v .h$$
\subsection{Densité linéique}
On associe à $\vec j_v$ un courant linéique d'intensité I.
\newpage
\section{Origine du champ magnétostatique}
\subsection{Loi de Biot et Savart}
$$\vec dB(M) = {\mu_0 \over 4\pi}{I \vec dl \wedge \vec PM \over PM^3}$$
Remarque : \\ \\
Avec $\vec j(P)$ on remplace $I\vec dl$ par $\vec j(P)d\tau$.\\
Avec $\vec j_s(P)$ on remplace $I\vec dl$ par $\vec j_sdS$. \\
Si les courants sont confinés dans le volume $\Omega$, le champ magnétique total est la somme des champs magnétiques élémentaires.\\
On applique donc le principe de superposition : 
$$ex~:~~\vec B(M) = \iiint_\Omega {\mu_0 \over 4\pi}\vec j(P){d\tau \wedge \vec PM \over PM^3}$$
Remarque :
$${\vec PM \over PM^3} = {\vec PM \over PM}{1 \over PM^2} = \vec u_{PM}.{1 \over PM^2}$$
$\mu_0$ est la perméabilité magnétique du vide. \\
Dans le système internationnal : $\mu_0 = 4\pi.10^{-7} H.M^{-1}$.
\newpage
\section{Propriétés de symétrie}
\begin{itemize}
\item Par rapport à un plan de symétrie
\begin{itemize}
\item la composante de $\vec B(M)$ parallèle à ce plan est changée en son opposée.
\item la composante de $\vec B(M)$ perpendiculaire à ce plan est conservée $\vec B(M) = \vec B(M')$.
\item Si M appartient à un plan de symétrie de courant, $\vec B(M)$ est perpendiculaire à ce plan.
\end{itemize}
\item Par rapport à un plan d'anti-symétrie
\begin{itemize}
\item la composante de $\vec B(M)$ parallèle à ce plan est conservée.
\item la composante de $\vec B(M)$ perpendiculaire à ce plan est changée en son opposée.
\item Si M appartient à un plan d'anti-symétrie de courant, $\vec B(M)$ est parallèle à ce plan.
\item Si M apartient à deux plans d'anti-symétrie, $\vec B(M)$ a pour direction celle de la droite intersection des deux plans.
\end{itemize}
\end{itemize}
Ayant la direction de $\vec B(M)$, les invariances par rotation ou par translation (de la distribution de courant) permettent d'éliminer des variables. On évite des calculs.
\newpage
\section{Le théorème d'Ampère}
Énoncé : \\ \\
La circulation  de $\vec B(M)$ sur un contour fermé $\Gamma$ est égal à $\mu_0$ fois l'intensité du courant qui traverse une surface $\Sigma$ quelconque s'appuyant sur $\Gamma$.
$$\int_\Gamma \vec B(M \in \Gamma).\vec dl = \mu_0 I = \mu_0 \iint_\Sigma \vec j(M \in \Gamma).\vec dS$$
Si on a déterminé la direction de $\vec B(M)$ et les variables dont il dépend, on essaie de trouver un contour $\Gamma$ sur lequel $\vec B(M).\vec dl = B(M).dl$ au moins sur une partie. \\ \\
Ex : Un fil dans lequel circule I, le fil est parallèle à Oz et il est infini.
\begin{itemize}
\item Point 1 ; Direction de $\vec B(M)$ ?
\begin{itemize}
\item Tout plan défini par ($\vec u_\rho, \vec u_\phi$) est plan de symétrie pour la distribution de courant. Pour tous M de l'espace,  il appartient forcément à un de ces plans : \\
$\vec B(M)$ est perpendiculaire au plan de symétrie : $\vec B(M) = B(M) \vec u_\phi$.
\item Invariances ?
\begin{itemize}
\item Oui par translation le long de $Oz \rightarrow B(M) = B(\rho, \phi)$
\item Oui par rotation autour de $Oz \rightarrow B(M) = B(\rho)$
$$\vec B = B(\rho)\vec u_\phi$$
\end{itemize}
\end{itemize}
\item Point 2 : défini par $\Gamma$, contour fermé
\begin{itemize}
\item $\Gamma$ : cercle d'axe Oz, de rayon $\rho = OM$
$$dC_\Gamma = \vec B(M \in \Gamma).\vec dl$$
$$dC_\Gamma = B(\rho)\vec u_\phi . dl \vec u_\phi = B(\rho)dl$$
$$C_\Gamma = \int_\Gamma \vec B.\vec dl = \int_\Gamma B(\rho)dl = B(\rho)\int_\Gamma dl = 2\pi \rho B(\rho)$$
Application au théorème d'Ampère : 
$$2\pi\rho B(\rho)=\mu_0 \iint_{\Sigma sur \Gamma}\vec j . \vec dS$$
\end{itemize}
\end{itemize}
\end{document}